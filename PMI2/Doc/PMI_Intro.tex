% =======================================================================
%  PMI Toolbox Documentation
%  Copyright (C) 2004  Jonathan Stott
%  
%  This program is free software; you can redistribute it and/or
%  modify it under the terms of the GNU General Public License
%  as published by the Free Software Foundation; either version 2
%  of the License, or (at your option) any later version.
%  
%  This program is distributed in the hope that it will be useful,
%  but WITHOUT ANY WARRANTY; without even the implied warranty of
%  MERCHANTABILITY or FITNESS FOR A PARTICULAR PURPOSE.  See the
%  GNU General Public License for more details.
%  
%  You should have received a copy of the GNU General Public License
%  along with this program; if not, write to the Free Software
%  Foundation, Inc., 59 Temple Place - Suite 330, 
%  Boston, MA  02111-1307, USA.
% =======================================================================

\documentclass[12pt]{article}

%%%%%%%%%%%%%%%%%%%%%%%%%%%%%%%%%%%%%%%%%%%%%%%%%%%%%%%%%%%%%%%%%%%%%%%%
% Tweak default margins.

\setlength{\evensidemargin}{0pt}
\setlength{\oddsidemargin}{0pt}
\setlength{\rightmargin}{0pt}
\setlength{\leftmargin}{0pt}

\setlength{\topmargin}{-0.25in}		% Tweak the spacings
\setlength{\headheight}{\baselineskip}
\addtolength{\headheight}{4pt}         \addtolength{\topmargin}{-\headheight}
\setlength{\headsep}{\baselineskip}    \addtolength{\topmargin}{-\baselineskip}
\setlength{\footskip}{2\baselineskip}

% Recompute page size using current margin settings

\setlength{\textwidth}{\paperwidth}
\addtolength{\textwidth}{-2.0in}		% Implicit margins
\addtolength{\textwidth}{-2\oddsidemargin}	% Explicit margins

\setlength{\textheight}{\paperheight}
\addtolength{\textheight}{-2.0in}		% Implicit margins
\addtolength{\textheight}{-\headheight}		% Explicit margins
\addtolength{\textheight}{-\headsep}
\addtolength{\textheight}{-\topmargin}

\addtolength{\textheight}{0.25in} % Excess on bottom
%%%%%%%%%%%%%%%%%%%%%%%%%%%%%%%%%%%%%%%%%%%%%%%%%%%%%%%%%%%%%%%%%%%%%%%%

\newcommand{\pmifield}[1]{{\tt #1}}
\newcommand{\pmifile}[1]{{\tt #1}}

% Check for PDFLaTeX
\newif\ifpdf
\ifx\pdfoutput\undefined
  \pdffalse    % we are not running PDFLaTeX
\else
  \pdfoutput=1 % we are running PDFLaTeX
  \pdftrue
\fi

\ifpdf
  % Set PDF document options
  \pdfcompresslevel 6
  \pdfinfo { 
    /Title (PMI_Intro.pdf) 
    /Creator (pdfLaTeX) 
    /Producer (pdfLaTeX 3.14159-1.10b) 
    /Author (Jonathan Stott) 
    /CreationDate (D:20030314164306) 
    /ModDate (D:20040113134800) 
    /Subject (Introduction to the PMI Toolbox) 
    /Keywords (PMI,DOT,Matlab) }
\fi

\begin{document}

\title{PMI Toolbox Documentation}
\author{Jonathan Stott}
% \address{Martinos Center for Biomedical Imaging, Massachusetts General
% Hospital}
\date{\today}

\maketitle

\begin{abstract}
This file has been superceeded by the HTML documentation.
\end{abstract}

\section{PMI Structures}

Information about the imager configuration is stored in the \pmifield{SD}
structure.  Information about the volume being imaged is stored in the
\pmifield{Medium} structure.  The fields used by each are defined below.
Inappropriate fields (e.g., \pmifield{SD.ModFreq} for a time-domain
system) generally do not need to be defined.

\subsection{SD Structure}

The fields in the \pmifield{SD} structure and their meaning:
\begin{center}
\begin{tabular}{|c|l|}
\hline
\pmifield{Lambda}        & Measurement wavelengths (nm) \\[2pt]
\pmifield{ModFreq}       & Modulation Frequency (MHz) \\[2pt]
\pmifield{TimeDelay}     & Delay to start of time gate (sec) \\[2pt]
\pmifield{TimeGateWidth} & Width of time gate (sec) \\[2pt]
\hline
\pmifield{SrcPos}        & Source Positions (cm) [$N_{src}\times3$] \\[2pt]
\pmifield{SrcAmp}        & Complex Source Amplitude 
				[$N_{src}\times N_{\lambda}\times N_{freq}$] 
				[-] \\[2pt]
\pmifield{SrcOffset}     & Source Temporal Offset (sec) \\[2pt]
\hline
\pmifield{DetPos}        & Detector Positions (cm) [$N_{src}\times3$] \\[2pt]
\pmifield{DetAmp}        & Complex Detector Amplitude 
				[$N_{src}\times N_{\lambda}\times N_{freq}$] 
				[-] \\[2pt]
\pmifield{DetOffset}     & Detector Temporal Offset (sec) \\[2pt]
\hline
\pmifield{MeasList}      & Measurement List (see below) \\[2pt]
\hline
\end{tabular}
\end{center}

\subsubsection{Measurement List}

The measurement list \pmifield{SD.MeasList} is a $N_{meas}\times
N_{fields}$ table of indices into the other fields in SD (and,
indirectly, Medium) on a measurement by measurement basis.  Currently
there are 9 fields in the measurement list (although not all of them
are supported by the toolbox yet).  The nine fields are:
\begin{enumerate}
\item \pmifield{SrcPos}
\item \pmifield{DetPos}
\item \pmifield{ModFreq}
\item \pmifield{Lambda}
\item \pmifield{EmissionWavelength} (not supported yet)
\item \pmifield{TimeDelay}
\item \pmifield{TimeGateWidth}
\item \pmifield{CorrelationTime} (not supported yet)
\item \pmifield{DataFormat} (not supported yet)
\end{enumerate}
The number of fields is not fixed; additional fields will be added in
the future as the need arises.  The order of the fields, however,
will not change.

Indicies to fields that have no meaning in the context of the
measurement (e.g., \pmifield{SD.ModFreq} with a time domain imager)
should be set to `0' and the corresponding field does not need to be
included in \pmifield{SD}.  Indicies into fields that do exist (e.g.,
\pmifield{SD.SrcPos}, which always applies) should be between $1$ and
$N_f$ where $N_f$ is the number of elements in the appropriate table
(e.g., back to our \pmifield{SD.SrcPos} example, if there are $5$
sources defined then the largest allowed value of
\pmifield{SD.MeasList(:,1)} is $5$).

\pmifield{readPMIData.m} generates \pmifield{SD.MeasList} as it loads
the data; \pmifield{genMeasList} generates \pmifield{SD.MeasList} for
simulated data.  The measurement list can also be generated manually,
if needed.  The order of fields is not sigificant, but it must
match the experimental data (if any).

\subsection{Medium Structure}

\pmifield{Medium} describes the optical properties of the medium and
describes the volume to be reconstructed.  The fields in
\pmifield{Medium} are: 
\begin{center}
\begin{tabular}{|c|l|}
\hline
\pmifield{Musao}          & Absorption Coefficient (1/cm) [$1\times N_{wvl}$]\\
\pmifield{Muspo}          & Transport Scattering Coefficient 
				(1/cm) [$1\times N_{wvl}$]\\
\pmifield{idxRefr}        & Index of Refraction [$1\times N_{wvl}$]\\
\hline
\pmifield{Geometry}       & Model geometry (`inf', `semi', or `slab')\\
\pmifield{Slab}\_Thickness& Slab thickness, if applicable \\
\pmifield{CompVol}        & Computational region of interest (see below) \\
\hline
\pmifield{Object}         & Optical perturbations in ROI (see below) \\
\hline
\end{tabular}
\end{center}

\subsubsection{CompVol Structure}

\pmifield{Medium.CompVol} specifies the computational region of
interest (i.e., the volume to be reconstructed).  There are several
ways of specifying the volume, with varying degrees of support within
the toolbox.  When in doubt, use ``uniform''.  Alternate types
are ``computed'' and ``list''---see sampleVolume.m for how they're used.

\begin{center}
\begin{tabular}{|c|l|}
\hline
\pmifield{Type}  & How volume is specified, `uniform' is recommended \\
\hline
\pmifield{X}     & Points on X axis (uniformly spaced if type `uniform') \\
\pmifield{Y}     & Points on Y axis \\
\pmifield{Z}     & Points on Z axis \\
\hline
\pmifield{XStep} & \\
\pmifield{YStep} & Voxel size, used to compute dV \\
\pmifield{ZStep} & \\
\hline
\end{tabular}
\end{center}

\subsubsection{Object Vector}

\pmifield{Medium.Object} is a vector of cells that specify
perturbations used in calculating the forward problem.  Object does not
need to be defined if you're working with homogeneous volumes.

All \pmifield{Object}'s share three fields.  
\begin{center}
\begin{tabular}{|c|l|}
\hline
Type & Type of perturbation \\
\hline
Mua  & Absorption (absolute) in the perturbation, per wavelength \\
Musp & Scattering (absolute) in the perturbation, per wavelength \\
\hline
\end{tabular}
\end{center}

In addition, depending on the value of
\pmifield{Object.Type}, additional fields may need to be defined.
\begin{center}
\begin{tabular}{|c|l|l|}
\hline
``Sphere'' & \pmifield{Object.Pos}    & Center of sphere (X,Y,Z) \\
           & \pmifield{Object.Radius} & Radius of sphere (cm) \\
\hline
``Block''  & \pmifield{Object.Pos}    & Center of box (X,Y,Z) \\
           & \pmifield{Object.Dims}   & Length of edges (cm) \\
\hline
``Image''  & \multicolumn{2}{l}{No extra fields, but
		\pmifield{Mua} and \pmifield{Musp} are
		 $N_{voxel}\times N_\lambda$} \\
\hline
\end{tabular}
\end{center}
See calcDelMuA.m for complete details.

\section{PMI Toolbox routines}

This is a mostly-complete list of toolbox routines and what they do.
Utility routines that are not normally called by the end user are not
included.  In all cases, ``help function\_name'' should print out some
useful information on how to call the individual function
\pmifile{function\_name}.

\subsection{General utilities}

\begin{tabular}{|l|l|}
\hline
\pmifile{pmipath.m}      & Add all the PMI toolbox directories to the
				current path \\ \hline 
\pmifile{genMeasList.m}  & Auto-generate a measurement list \\ \hline
\pmifile{sampleVolume.m} & Given Medium.CompVol, generate a vector of
				voxel centers \\ \hline
\pmifile{calcSep.m}      & Calculate source-detector separation of each
				measurement pair \\ \hline 
\pmifile{calcAffine.m}   & Calculate Affine transform between two sets of
				coordinates \\ \hline
\end{tabular}

\subsection{Data processing}

\begin{tabular}{|l|l|}
\hline
\pmifile{readPMIData.m}   & Read PMI Data Formatted files \\ \hline
\pmifile{writePMIData.m}  & Write data out in PMI Data Format \\ \hline
\pmifile{fitSD.m}         & Fit complex source-detector coefficients \\ \hline
\pmifile{fitSDAmp.m}      & Fit source-detector, amplitude only \\ \hline
\pmifile{fitSDPhs.m}      & Fit source-detector, phase only \\ \hline
\pmifile{fitBackground.m} & Brute-force search for optical properties \\ \hline
\end{tabular}

\subsubsection{Visualization}

\begin{tabular}{|l|l|}
\hline
\pmifile{plotData.m}       & Plot data seen by each source/detector \\ \hline
\pmifile{showImage.m}      & Show reconstructed data as slices \\ \hline
\pmifile{showVolume.m}     & Show reconstructed data as volume \\ \hline
\pmifile{plotDimensions.m} & Aesthetic subplot sizes \\ \hline
\end{tabular}

\subsection{Forward problem}

\subsubsection{General}

\begin{tabular}{|l|l|}
\hline
\pmifile{isFD.m}                & Given SD, decide if 
		FD is appropriate \\ \hline
\pmifile{isTD.m}                & Given SD, decide if 
		TD is appropriate \\ \hline
\pmifile{calcDelMuA.m}          & Wrapper, calculate abs. 
		perturbations \\ \hline
\pmifile{calcDelMuSp.m}         & Wrapper, calculate scat. 
		perturbations \\ \hline
\pmifile{genBlock.m}            & Generate block perturbation \\ \hline
\pmifile{genSphere.m}           & Generate spherical Jacobian \\ \hline
\pmifile{Pert\_Grad.m}          & Gradient of a perturbation \\ \hline
\pmifile{genSDJacobian.m}       & Jacobian for amplitude update \\ \hline
\pmifile{genPosJacobian.m}      & Jacobian for position update \\ \hline
\pmifile{genMuJacobian.m}       & Jacobian for $\mu$ \\ \hline
\pmifile{genJacobianFrom2pts.m} & Jacobian from full-born calc \\ \hline
\pmifile{DPDWHelmholtz.m}       & High-level wrapper, generate 2-pt\\ \hline
\pmifile{genBornMat.m}          & Calculate forward problem $A$ and
					$\phi_0$ \\ \hline 
\pmifile{genBornData.m}         & Calculate forward problem and
					$\phi_{scat}$ \\ \hline 
\pmifile{calcExtBnd.m}          & Calculate extrapolated boundary \\ \hline
\pmifile{getImageCharge.m}      & Given $z_{bnd}$, find $z_{img}$ \\ \hline
\pmifile{moveSrcSlab.m} 	& In a slab or semi-infinite geometry, move 
					``sources'' one \\
                		& \quad scattering length 
					into the medium \\ \hline 
\end{tabular}

\subsubsection{Frequency domain/Continuous wave imaging}

\begin{tabular}{|l|l|}
\hline
\pmifile{FD2pt.m}           & Generate FD fluence $\phi_0$ \\ \hline
\pmifile{FD3pt.m}           & Generate FD sensitivity matrix $A$ \\ \hline
\pmifile{FD\_GF.m}          & FD Green's function\\ \hline
\pmifile{FD\_GradGF.m}      & Gradient of FD Green's function\\ \hline
\end{tabular}

\subsubsection{Time domain imaging}

\begin{tabular}{|l|l|}
\hline
\pmifile{TD2pt.m}      & Generate TD fluence $\phi_0$ \\ \hline
\pmifile{TD3pt.m}      & Generate TD sensitivity matrix $A$ \\ \hline
\pmifile{TD\_GF.m}     & TD Green's function \\ \hline
\pmifile{TD\_GradGF.m} & Gradient of TD Green's function \\ \hline
\end{tabular}

\subsection{Interfacing with tMCimg and tFDimg forward modelers}

\begin{tabular}{|l|l|}
\hline
\pmifile{PMItoMC.m}	& Given \pmifield{SD}/\pmifield{Medium}, write
			out .cfg config file for \pmifile{tMCimg} \\ \hline
\pmifile{PMItoFD.m}	& Given \pmifield{SD}/\pmifield{Medium}, write
			out .cfg config file for \pmifile{tFDimg} \\ \hline
\pmifile{MCConfig.m}	& Read a .inp config file \\ \hline
\pmifile{loadMC2pt.m}	& Read a .2pt photon density file \\ \hline
\pmifile{readMCHis.m}	& Read a .his photon history file \\ \hline
\end{tabular}

\subsection{Inverse problem}

Most direct linear reconstructions (e.g.\ Tikhonov, tSVD, etc.) are
simple enough to be handled directly on the command line, so there's
no need to write a separate inversion routine.  The routines below,
however, are sufficiently involved to warrant writing up as individual
functions.

\noindent
\begin{tabular}{|l|l|}
\hline
\pmifile{art.m}   & Algebraic Reconstruction Technique \\ \hline
\pmifile{sirt.m}  & Simultaneous Iterative Reconstruction Technique \\ \hline
\pmifile{tcgls.m} & Truncated Conjugate Gradient \\ \hline
\end{tabular}

\end{document}
