% =======================================================================
%  PMI Toolbox Documentation
%  Copyright (C) 2004  Jonathan Stott
%  
%  This program is free software; you can redistribute it and/or
%  modify it under the terms of the GNU General Public License
%  as published by the Free Software Foundation; either version 2
%  of the License, or (at your option) any later version.
%  
%  This program is distributed in the hope that it will be useful,
%  but WITHOUT ANY WARRANTY; without even the implied warranty of
%  MERCHANTABILITY or FITNESS FOR A PARTICULAR PURPOSE.  See the
%  GNU General Public License for more details.
%  
%  You should have received a copy of the GNU General Public License
%  along with this program; if not, write to the Free Software
%  Foundation, Inc., 59 Temple Place - Suite 330, 
%  Boston, MA  02111-1307, USA.
% =======================================================================

\documentclass[12pt]{article}

%%%%%%%%%%%%%%%%%%%%%%%%%%%%%%%%%%%%%%%%%%%%%%%%%%%%%%%%%%%%%%%%%%%%%%%%
% Tweak default margins.

\setlength{\evensidemargin}{0pt}
\setlength{\oddsidemargin}{0pt}
\setlength{\rightmargin}{0pt}
\setlength{\leftmargin}{0pt}

\setlength{\topmargin}{-0.25in}		% Tweak the spacings
\setlength{\headheight}{\baselineskip}
\addtolength{\headheight}{4pt}         \addtolength{\topmargin}{-\headheight}
\setlength{\headsep}{\baselineskip}    \addtolength{\topmargin}{-\baselineskip}
\setlength{\footskip}{2\baselineskip}

% Recompute page size using current margin settings

\setlength{\textwidth}{\paperwidth}
\addtolength{\textwidth}{-2.0in}		% Implicit margins
\addtolength{\textwidth}{-2\oddsidemargin}	% Explicit margins

\setlength{\textheight}{\paperheight}
\addtolength{\textheight}{-2.0in}		% Implicit margins
\addtolength{\textheight}{-\headheight}		% Explicit margins
\addtolength{\textheight}{-\headsep}
\addtolength{\textheight}{-\topmargin}

\addtolength{\textheight}{0.25in} % Excess on bottom
%%%%%%%%%%%%%%%%%%%%%%%%%%%%%%%%%%%%%%%%%%%%%%%%%%%%%%%%%%%%%%%%%%%%%%%%

% Check for PDFLaTeX
\newif\ifpdf
\ifx\pdfoutput\undefined
  \pdffalse    % we are not running PDFLaTeX
\else
  \pdfoutput=1 % we are running PDFLaTeX
  \pdftrue
\fi

\ifpdf
  % Set PDF document options
  \pdfcompresslevel 6

  \pdfinfo { 
    /Title (PMIDataFormat.pdf) 
    /Creator (pdfLaTeX) 
    /Producer (pdfLaTeX 3.14159-1.10b) 
    /Author (Jonathan Stott)
    /CreationDate (D:20020715143006) 
    /ModDate (D:20040319144600) 
    /Subject (Introduction to the PMI Data Format) 
    /Keywords (PMI,DOT,Matlab,datafile format) }
\fi

\newcommand{\keyword}[1]{\mbox{$\left<\mbox{\it #1}\right>$}\/}

\begin{document}

\title{PMI Data File Format}
\author{Jonathan Stott}
% \address{Martinos Center for Biomedical Imaging, Massachusetts General
% Hospital}
\date{\today}

\maketitle

This file documents the data file format used in V2.0 of the PMI
toolbox.  Because it includes details about the experimental geometry
such as source and detector positions, this is the preferred format
for archiving instrument data.

The general format of the data file consists of two parts: a text
header followed by the binary data (see Figure~\ref{fig:sample_file}
for an example).  The text header is parsed line by
line.  Each line contains a single $\keyword{keyword}=\keyword{value}$
pair.  Blank lines are ignored.  Comments are introduced by the `\%'
character and continue to the end of the current line.

\begin{figure}[p]
\begin{verbatim}
%%%%%%%%%%%%%%%%%%%%%%%%%%%%%%%%%%%%%%%%%%%%%%%%%%%%%%%%%%%%
% Data generated by the WhizBang CW Imager
%%%%%%%%%%%%%%%%%%%%%%%%%%%%%%%%%%%%%%%%%%%%%%%%%%%%%%%%%%%%

DetPos(1) = [  10  10   0 ]        % Four detectors
DetPos(2) = [  10 -10   0 ]
DetPos(3) = [ -10 -10   0 ]
DetPos(4) = [ -10  10   0 ]

SrcPos(1) = [   0   0   0 ]       % One Source

Lambda(1) = 690     % Two wavelengths
Lambda(2) = 830

Frequency = 0

DataPrecision = 'unsigned short'; % 16-bit data
DataType(1)   = { 'Amplitude' };  % data is an amplitude

% Each frame of binary data has eight measurements
Meas(1) = [ 1 1 1 ]
Meas(2) = [ 1 2 1 ]
Meas(3) = [ 1 3 1 ]
Meas(4) = [ 1 4 1 ]
Meas(5) = [ 1 1 2 ]
Meas(6) = [ 1 2 2 ]
Meas(7) = [ 1 3 2 ]
Meas(8) = [ 1 4 2 ]

% Begin experimental data, must contain an integer
%  number of frames of data
BeginData
\end{verbatim}
\keyword{Sequential frames binary data}
\caption{Sample PMI data file}
\label{fig:sample_file}
\end{figure}

%%%%%%%%%%%%%%%%%%%%%%%%%%%%%%%%%%%%%%%%%%%%%%%%%%%%%%%%%%%%%%%%%%%%%%%%

\section{Text Keywords}
\label{sec:keywords}

There are a number of keywords that define the imager parameters and
the experimental geometry.  See Table~\ref{tbl:keywords} for a
complete list of keywords.  PMI structured data files are read into
the PMI toolbox using the Matlab function {\tt readPMIData()}.  This
function returns a PMI structure initialized using the information in
the text header and a matrix containing the data in the binary
portion.  Most (but, unfortunately, not all) the keywords were
selected to have the same name and meaning as the fields within the
PMI structure.

%%%%%%%%%%%%%%%%%%%%%%%%%%%%%%%%%%%%%%%%%%%%%%%%%%%%%%%%%%%%%%%%%%%%%%%%

\begin{table}[p]
\begin{center}
\begin{tabular} {|c|l|}
\hline
Keyword & Description \\ \hline
\hline
\% \ldots & Comment---ignored \\ \hline
SrcPos               & Source position  \\ \hline
DetPos               & Detector position \\ \hline
ModFreq              & Source modulation frequency \\ \hline
Lambda               & Input wavelength \\ \hline
ExcitationWavelength & Input wavelength (for fluorescence) \\ \hline
EmissionWavelength   & Output wavelength (for fluorescence) \\ \hline
TimeDelay            & Electronic trigger delay \\ \hline
TimeGateWidth        & Width of the time gate   \\ \hline
CorrelationTime      & Correlation time         \\ \hline
ImagerOption         & Support for non-standard extensions \\ \hline
Meas                 & Declare measurement \\ \hline

DataPrecision        & Data format    \\ \hline
DataType             & Data semantics \\ \hline
BeginData            & End text header, begin binary data \\ \hline
\end{tabular}
\end{center}
\caption{Keywords used in the header portion of the PMI data file.}
\label{tbl:keywords}
\end{table}

%%%%%%%%%%%%%%%%%%%%%%%%%%%%%%%%%%%%%%%%%%%%%%%%%%%%%%%%%%%%%%%%%%%%%%%%

\subsection{Imaging Parameters}
\label{sub:imgparam}

Each of the imaging keywords (defined below) require an index number
as part of the keyword declaration (e.g.\ {\tt SrcPos(1)} declares the
location of source one).  If there is exactly one possible value, the
index may be left off ({\tt SrcPos}) and a value of `1' will
implicitly be used.  Imaging parameters may be declared in any order,
as long as they are defined before being used in a measurement.  If an
index is used more than once, the last declaration seen by the parser
will be used.  If an index is not used, the behavior of the parser is
undefined.

Keywords that are not applicable to a specific imager should not be
specified in the header.  There are {\bf no} default values; fields
not mentioned in the header are left undefined.

\subsubsection{Fiber Locations}

The \keyword{SrcPos} and \keyword{DetPos} keywords define the position
of the source and detector optodes respectively.  Their argument is
the $\left[x,y,z\right]$ position of the optode (including the
surrounding square brackets; the separating commas are optional).  For
example,
\[
\mbox{\tt SrcPos(1) = [ 0, 0, 0 ]} 
\]
declares the first source and defines it's located to be the origin of
the coordinate system.

\subsubsection{Imager configuration}

The \keyword{ModFreq} keyword defines the source modulation frequency for
frequency-domain imaging.  Its value should be given in Megahertz.

The \keyword{ExcitationWavelength} and \keyword{EmissionWavelength}
keywords define the wavelength of the light used for flourscence imaging.
Their value should be given in nano\-meters.  \keyword{Lambda} is used
for non-fluorescence image.  Currently, \keyword{Lambda} and
\keyword{ExcitationWavelength} are completely equivalent.  This could
change in the future, though, so please use whichever keyword is
appropriate for your specific application.

The keywords \keyword{TimeDelay}, \keyword{TimeGateWidth}, and
\keyword{CorrelationTime} are used with time-domain imagers.  Their
values should be specified in seconds.  \keyword{TimeDelay} specifies
the electronic delay introduced between the source pulse and the start
of data collection.  \keyword{TimeGateWidth} specifies either the gate width
of a gated time-domain imager or the bin width of a photon counting
system.  \keyword{CorrelationTime} is the correlation time for systems
that return that information.

\subsubsection{Other Configuration Parameters}

The \keyword{DataType} keyword is covered later in
Section~\ref{sub:dataparam}.

The final imaging parameter, \keyword{ImagerOption}, is special.  The
argument to \keyword{ImagerOption} is a text string enclosed in single
quotes and surrounded by curly brackets $\{\}$.
\keyword{ImagerOption} is stored in the PMI structure as a text
string---no attempt is made to assign any special meaning to the
argument.  The intended use of \keyword{ImagerOption} is two-fold.
First, it provides a stop-gap means to record essential imager data
until the PMI toolbox can be extended with the appropriate fields.
Second, it provides a means for the imager to include information that
is important enough to be worth record, but will not be used to
calculate the forward problem (e.g., sampling rate for time-course
data).

%%%%%%%%%%%%%%%%%%%%%%%%%%%%%%%%%%%%%%%%%%%%%%%%%%%%%%%%%%%%%%%%%%%%%%%%

\subsection{Measurement Lists}
\label{sub:measlist}

The measurement list holds the mapping between the individual data
elements and the various imaging parameters
(Section~\ref{sub:imgparam}).  Every data element {\bf must} have a
corresponding \keyword{Meas} declaration.  While measurements and
imager option may be interleaved, we recommend placing measurements at
the end of the file (after the imager parameters) for the sake of
clarity.

Each measurement is declared using the \keyword{Meas} keyword; its
argument is a list of indices into the imager parameters, surrounded by
square brackets, e.g.
\[
\mbox{\tt Meas(3) = [ 1 3 5 ].}
\]
The first two fields (with values of `1' and `3' in the example above)
are always the index of the source and detector respectively.  The
meaning of the remaining fields depends on which of the imaging
parameters are defined (hence the recommendation to put the
measurements after the imaging parameters).  {\it Imaging parameters
(except source and detector index) are not included in the measurement
list unless they can take at least two possible values}.  In the
example file in Figure~\ref{fig:sample_file}, the three fields in the
measurement list are \keyword{SrcPos}, \keyword{DetPos}, and
\keyword{Lambda}.  \keyword{ModFreq} is not a field in the
measurement list because it only has a single defined value.

While the do not need to be defined in order, there must not be any
gaps in the final list of measurements.  That is, if {\tt Meas(10)} is
declared in the file, measurements {\tt Meas(1)} through {\tt Meas(9)}
must also be declared somewhere in the file.  The behavior of the
toolbox when presented with an incomplete measurement list is
undefined.

The order of the fields is fixed (see Table~\ref{tbl:mlorder}).  After
reading the entire text header, the parser pads out the given
measurements into a full measurement list.  Undeclared fields are
assigned a value of `0'.  Fields that take a single value are
implicitly set to `1'.

\begin{table}
\begin{center}
\begin{tabular}{|c|l|}
\hline
Field \# & Index Into \\ \hline
\hline
1 & Source Position       \\ \hline
2 & Detector Position     \\ \hline
3 & Modulation Frequency  \\ \hline
4 & Source  Wavelength    \\ \hline
5 & Emission Wavelength   \\ \hline
6 & Delay Time            \\ \hline
7 & Gate Width            \\ \hline
8 & Correlation Time      \\ \hline
9 & Data Type             \\ \hline
\end{tabular}
\end{center}
\caption{Ordering of fields within a measurement declaration.  Source
and detector fields are always required.  The remaining fields should
be deleted if either they are not relevant (no defined values) or
there is only one possible value. Fields with no defined values will
be implicitly set to '0', fields with only one possible value will be
implicitly set to '1'.}
\label{tbl:mlorder}
\end{table}

%%%%%%%%%%%%%%%%%%%%%%%%%%%%%%%%%%%%%%%%%%%%%%%%%%%%%%%%%%%%%%%%%%%%%%%%

\subsection{Data Parameters}
\label{sub:dataparam}

There are three paramters that define the formatting of the data in the
binary portion of the file.  The first, \keyword{DataPrecision},
specifies the binary format of the data.  It's value is any of the
precision strings understood by Matlab (for example `float32'; type
{\tt help fread} to get a complete list of precision strings).  All
measurements are assumed to have the same precision.  The default data
precision is `float32', which corresponds to C's builtin ``float''
data type.

The second data paramter keyword is \keyword{DataType} which gives the
symantic content of the data.  That is to say, \keyword{DataType}
tells the toolbox what meaning should be assinged to data as it is
read in.  Similar to \keyword{ImagerOption} in
Section~\ref{sub:imgparam}, \keyword{DataType} is followed by an index
in parenthesis and its argument is enclosed in single quotes,
surrounded by a pair of curly brackets.
{\bf This option is not yet implemented by the toolbox}.
\keyword{DataType} does not have a default value.  Possible arguments
to \keyword{DataType} are given in Table~\ref{tbl:datatype}.

The final keyword is \keyword{BeginData}.  This keyword takes no
arguments and signals the parser that it has reached the end of the
text header.  The data portion begins immediately after the closing
newline.

\begin{table}
\begin{center}
\begin{tabular}{|c|l|}
\hline
Type & Description \\ \hline
\hline
Amplitude   & Amplitude data \\ \hline
Phase       & Phase data \\ \hline
I           & Real part of signal \\ \hline
Q           & Imaginary part of signal \\ \hline
IQ          & Complex storage (real,imaginary) \\ \hline
Real        & Same as `I' \\ \hline
Imaginary   & Same as `Q' \\ \hline
Complex     & Same as 'IQ' \\ \hline
AmpStdErr   & Standard deviation of amplitdue data \\ \hline
PhaseStdErr & Standard deviation of phase data \\ \hline
IQStdErr    & Standard devaition of I/Q data \\ \hline
\end{tabular}
\end{center}
\caption{Possible values of \keyword{DataType}}
\label{tbl:datatype}
\end{table}

%%%%%%%%%%%%%%%%%%%%%%%%%%%%%%%%%%%%%%%%%%%%%%%%%%%%%%%%%%%%%%%%%%%%%%%%

\section{Binary Data}

Data is stored in the binary portion of the file in one or more
frames.  A frame is defined as a complete set of measurements, as
specified using \keyword{Meas}.  The size of each data element is
given by \keyword{DataPrecision}.  Frames are read sequentially; the
data portion of the file must contain an integer number of data
frames.  The data portion ends when the file ends.

\end{document}

